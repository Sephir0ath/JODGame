\documentclass{article}

\date{Octubre 2024}

\title{Informe}

\author{
	Juan Felipe Raysz Muñoz
	\and
	Oliver Isaías Peñailillo Sanzana
	\and
	Diego Emilio Rebollo García
}

\begin{document}
	\maketitle
	
	\section{Introducción}
		El proyecto consiste en desarrollar un videojuego 2D con vista aérea, el cual estará enfocado en un escenario donde el jugador deberá enfrentarse a enemigos o interactuar dinámicamente con los objetos del entorno para completar ciertos objetivos.
		
		El juego integra componentes de diseño de niveles, comportamiento dinámico de enemigos, uso de técnicas de raycasting para ampliar y mejorar la jugabilidad, entre otros.
		
		Una de las dimensiones más importantes del desarrollo del proyecto es propiciar una base sólida y modular para permitir futuras expansiones y mejoras en cualquier aspecto del juego.
	
	\section{Requisitos y especificaciones}
		El juego está siendo desarrollado en el lenguaje de programación C++ y hace uso de la librería SDL2 para el manejo del apartado gráfico.
		
		Los requisitos funcionales son los siguientes:
		
		\begin{enumerate}
			\item Seleccionar modo de juego
			\item Seleccionar ajustes
			\item Permitir al jugador desplazarse por el escenario
			\item Completar objetivos y avanzar de nivel
			\item Eliminar enemigos
		\end{enumerate}
	
	\section{Diseño}
		% Mostrar diagrama de casos de uso
	
	\section{Implementacion}
		\subsection{Jugador}
			\subsubsection{player.lookWalls()}
				Método que simula la acción de mirar hacia las murallas, calculando continuamente los puntos de intersección existentes con los objectos y trazando un rayo entre ambos.
			
			\subsubsection{playerController.handleInput()}
				Método que se encarga de manejar los inputs del teclado del jugador y realiza el movimiento.
		
		\subsection{Raycasting}
			\subsubsection{ray.cast()}
				Algoritmo que calcula el punto de intersección (si existe) entre dos segmentos de recta.
				
				Para ello, el algoritmo toma como parámetros los puntos extremos en cada segmento y obtiene el punto de intersección haciendo uso de la siguiente ecuación:
				
				% Imágen
	
	\section{Pruebas}
		% Puntos de interseccion en algoritmo de raycasting
		% Operaciones en la clase vector
	
	\section{Gestión del proyecto}
		%
	
	\section{Conclusiones y trabajos futuros}
		%
	
	\section{Anexo}
		Documentación de C++: https://en.cppreference.com/w/
		Documentación de SDL2: https://wiki.libsdl.org/SDL2/FrontPage
		Tutoriales de SDL2: https://lazyfoo.net/tutorials/SDL/index.php
\end{document}